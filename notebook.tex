
% Default to the notebook output style

    


% Inherit from the specified cell style.




    
\documentclass[11pt]{article}

    
    
    \usepackage[T1]{fontenc}
    % Nicer default font (+ math font) than Computer Modern for most use cases
    \usepackage{mathpazo}

    % Basic figure setup, for now with no caption control since it's done
    % automatically by Pandoc (which extracts ![](path) syntax from Markdown).
    \usepackage{graphicx}
    % We will generate all images so they have a width \maxwidth. This means
    % that they will get their normal width if they fit onto the page, but
    % are scaled down if they would overflow the margins.
    \makeatletter
    \def\maxwidth{\ifdim\Gin@nat@width>\linewidth\linewidth
    \else\Gin@nat@width\fi}
    \makeatother
    \let\Oldincludegraphics\includegraphics
    % Set max figure width to be 80% of text width, for now hardcoded.
    \renewcommand{\includegraphics}[1]{\Oldincludegraphics[width=.8\maxwidth]{#1}}
    % Ensure that by default, figures have no caption (until we provide a
    % proper Figure object with a Caption API and a way to capture that
    % in the conversion process - todo).
    \usepackage{caption}
    \DeclareCaptionLabelFormat{nolabel}{}
    \captionsetup{labelformat=nolabel}

    \usepackage{adjustbox} % Used to constrain images to a maximum size 
    \usepackage{xcolor} % Allow colors to be defined
    \usepackage{enumerate} % Needed for markdown enumerations to work
    \usepackage{geometry} % Used to adjust the document margins
    \usepackage{amsmath} % Equations
    \usepackage{amssymb} % Equations
    \usepackage{textcomp} % defines textquotesingle
    % Hack from http://tex.stackexchange.com/a/47451/13684:
    \AtBeginDocument{%
        \def\PYZsq{\textquotesingle}% Upright quotes in Pygmentized code
    }
    \usepackage{upquote} % Upright quotes for verbatim code
    \usepackage{eurosym} % defines \euro
    \usepackage[mathletters]{ucs} % Extended unicode (utf-8) support
    \usepackage[utf8x]{inputenc} % Allow utf-8 characters in the tex document
    \usepackage{fancyvrb} % verbatim replacement that allows latex
    \usepackage{grffile} % extends the file name processing of package graphics 
                         % to support a larger range 
    % The hyperref package gives us a pdf with properly built
    % internal navigation ('pdf bookmarks' for the table of contents,
    % internal cross-reference links, web links for URLs, etc.)
    \usepackage{hyperref}
    \usepackage{longtable} % longtable support required by pandoc >1.10
    \usepackage{booktabs}  % table support for pandoc > 1.12.2
    \usepackage[inline]{enumitem} % IRkernel/repr support (it uses the enumerate* environment)
    \usepackage[normalem]{ulem} % ulem is needed to support strikethroughs (\sout)
                                % normalem makes italics be italics, not underlines
    

    
    
    % Colors for the hyperref package
    \definecolor{urlcolor}{rgb}{0,.145,.698}
    \definecolor{linkcolor}{rgb}{.71,0.21,0.01}
    \definecolor{citecolor}{rgb}{.12,.54,.11}

    % ANSI colors
    \definecolor{ansi-black}{HTML}{3E424D}
    \definecolor{ansi-black-intense}{HTML}{282C36}
    \definecolor{ansi-red}{HTML}{E75C58}
    \definecolor{ansi-red-intense}{HTML}{B22B31}
    \definecolor{ansi-green}{HTML}{00A250}
    \definecolor{ansi-green-intense}{HTML}{007427}
    \definecolor{ansi-yellow}{HTML}{DDB62B}
    \definecolor{ansi-yellow-intense}{HTML}{B27D12}
    \definecolor{ansi-blue}{HTML}{208FFB}
    \definecolor{ansi-blue-intense}{HTML}{0065CA}
    \definecolor{ansi-magenta}{HTML}{D160C4}
    \definecolor{ansi-magenta-intense}{HTML}{A03196}
    \definecolor{ansi-cyan}{HTML}{60C6C8}
    \definecolor{ansi-cyan-intense}{HTML}{258F8F}
    \definecolor{ansi-white}{HTML}{C5C1B4}
    \definecolor{ansi-white-intense}{HTML}{A1A6B2}

    % commands and environments needed by pandoc snippets
    % extracted from the output of `pandoc -s`
    \providecommand{\tightlist}{%
      \setlength{\itemsep}{0pt}\setlength{\parskip}{0pt}}
    \DefineVerbatimEnvironment{Highlighting}{Verbatim}{commandchars=\\\{\}}
    % Add ',fontsize=\small' for more characters per line
    \newenvironment{Shaded}{}{}
    \newcommand{\KeywordTok}[1]{\textcolor[rgb]{0.00,0.44,0.13}{\textbf{{#1}}}}
    \newcommand{\DataTypeTok}[1]{\textcolor[rgb]{0.56,0.13,0.00}{{#1}}}
    \newcommand{\DecValTok}[1]{\textcolor[rgb]{0.25,0.63,0.44}{{#1}}}
    \newcommand{\BaseNTok}[1]{\textcolor[rgb]{0.25,0.63,0.44}{{#1}}}
    \newcommand{\FloatTok}[1]{\textcolor[rgb]{0.25,0.63,0.44}{{#1}}}
    \newcommand{\CharTok}[1]{\textcolor[rgb]{0.25,0.44,0.63}{{#1}}}
    \newcommand{\StringTok}[1]{\textcolor[rgb]{0.25,0.44,0.63}{{#1}}}
    \newcommand{\CommentTok}[1]{\textcolor[rgb]{0.38,0.63,0.69}{\textit{{#1}}}}
    \newcommand{\OtherTok}[1]{\textcolor[rgb]{0.00,0.44,0.13}{{#1}}}
    \newcommand{\AlertTok}[1]{\textcolor[rgb]{1.00,0.00,0.00}{\textbf{{#1}}}}
    \newcommand{\FunctionTok}[1]{\textcolor[rgb]{0.02,0.16,0.49}{{#1}}}
    \newcommand{\RegionMarkerTok}[1]{{#1}}
    \newcommand{\ErrorTok}[1]{\textcolor[rgb]{1.00,0.00,0.00}{\textbf{{#1}}}}
    \newcommand{\NormalTok}[1]{{#1}}
    
    % Additional commands for more recent versions of Pandoc
    \newcommand{\ConstantTok}[1]{\textcolor[rgb]{0.53,0.00,0.00}{{#1}}}
    \newcommand{\SpecialCharTok}[1]{\textcolor[rgb]{0.25,0.44,0.63}{{#1}}}
    \newcommand{\VerbatimStringTok}[1]{\textcolor[rgb]{0.25,0.44,0.63}{{#1}}}
    \newcommand{\SpecialStringTok}[1]{\textcolor[rgb]{0.73,0.40,0.53}{{#1}}}
    \newcommand{\ImportTok}[1]{{#1}}
    \newcommand{\DocumentationTok}[1]{\textcolor[rgb]{0.73,0.13,0.13}{\textit{{#1}}}}
    \newcommand{\AnnotationTok}[1]{\textcolor[rgb]{0.38,0.63,0.69}{\textbf{\textit{{#1}}}}}
    \newcommand{\CommentVarTok}[1]{\textcolor[rgb]{0.38,0.63,0.69}{\textbf{\textit{{#1}}}}}
    \newcommand{\VariableTok}[1]{\textcolor[rgb]{0.10,0.09,0.49}{{#1}}}
    \newcommand{\ControlFlowTok}[1]{\textcolor[rgb]{0.00,0.44,0.13}{\textbf{{#1}}}}
    \newcommand{\OperatorTok}[1]{\textcolor[rgb]{0.40,0.40,0.40}{{#1}}}
    \newcommand{\BuiltInTok}[1]{{#1}}
    \newcommand{\ExtensionTok}[1]{{#1}}
    \newcommand{\PreprocessorTok}[1]{\textcolor[rgb]{0.74,0.48,0.00}{{#1}}}
    \newcommand{\AttributeTok}[1]{\textcolor[rgb]{0.49,0.56,0.16}{{#1}}}
    \newcommand{\InformationTok}[1]{\textcolor[rgb]{0.38,0.63,0.69}{\textbf{\textit{{#1}}}}}
    \newcommand{\WarningTok}[1]{\textcolor[rgb]{0.38,0.63,0.69}{\textbf{\textit{{#1}}}}}
    
    
    % Define a nice break command that doesn't care if a line doesn't already
    % exist.
    \def\br{\hspace*{\fill} \\* }
    % Math Jax compatability definitions
    \def\gt{>}
    \def\lt{<}
    % Document parameters
    \title{A2 Road lane detection}
    
    
    

    % Pygments definitions
    
\makeatletter
\def\PY@reset{\let\PY@it=\relax \let\PY@bf=\relax%
    \let\PY@ul=\relax \let\PY@tc=\relax%
    \let\PY@bc=\relax \let\PY@ff=\relax}
\def\PY@tok#1{\csname PY@tok@#1\endcsname}
\def\PY@toks#1+{\ifx\relax#1\empty\else%
    \PY@tok{#1}\expandafter\PY@toks\fi}
\def\PY@do#1{\PY@bc{\PY@tc{\PY@ul{%
    \PY@it{\PY@bf{\PY@ff{#1}}}}}}}
\def\PY#1#2{\PY@reset\PY@toks#1+\relax+\PY@do{#2}}

\expandafter\def\csname PY@tok@w\endcsname{\def\PY@tc##1{\textcolor[rgb]{0.73,0.73,0.73}{##1}}}
\expandafter\def\csname PY@tok@c\endcsname{\let\PY@it=\textit\def\PY@tc##1{\textcolor[rgb]{0.25,0.50,0.50}{##1}}}
\expandafter\def\csname PY@tok@cp\endcsname{\def\PY@tc##1{\textcolor[rgb]{0.74,0.48,0.00}{##1}}}
\expandafter\def\csname PY@tok@k\endcsname{\let\PY@bf=\textbf\def\PY@tc##1{\textcolor[rgb]{0.00,0.50,0.00}{##1}}}
\expandafter\def\csname PY@tok@kp\endcsname{\def\PY@tc##1{\textcolor[rgb]{0.00,0.50,0.00}{##1}}}
\expandafter\def\csname PY@tok@kt\endcsname{\def\PY@tc##1{\textcolor[rgb]{0.69,0.00,0.25}{##1}}}
\expandafter\def\csname PY@tok@o\endcsname{\def\PY@tc##1{\textcolor[rgb]{0.40,0.40,0.40}{##1}}}
\expandafter\def\csname PY@tok@ow\endcsname{\let\PY@bf=\textbf\def\PY@tc##1{\textcolor[rgb]{0.67,0.13,1.00}{##1}}}
\expandafter\def\csname PY@tok@nb\endcsname{\def\PY@tc##1{\textcolor[rgb]{0.00,0.50,0.00}{##1}}}
\expandafter\def\csname PY@tok@nf\endcsname{\def\PY@tc##1{\textcolor[rgb]{0.00,0.00,1.00}{##1}}}
\expandafter\def\csname PY@tok@nc\endcsname{\let\PY@bf=\textbf\def\PY@tc##1{\textcolor[rgb]{0.00,0.00,1.00}{##1}}}
\expandafter\def\csname PY@tok@nn\endcsname{\let\PY@bf=\textbf\def\PY@tc##1{\textcolor[rgb]{0.00,0.00,1.00}{##1}}}
\expandafter\def\csname PY@tok@ne\endcsname{\let\PY@bf=\textbf\def\PY@tc##1{\textcolor[rgb]{0.82,0.25,0.23}{##1}}}
\expandafter\def\csname PY@tok@nv\endcsname{\def\PY@tc##1{\textcolor[rgb]{0.10,0.09,0.49}{##1}}}
\expandafter\def\csname PY@tok@no\endcsname{\def\PY@tc##1{\textcolor[rgb]{0.53,0.00,0.00}{##1}}}
\expandafter\def\csname PY@tok@nl\endcsname{\def\PY@tc##1{\textcolor[rgb]{0.63,0.63,0.00}{##1}}}
\expandafter\def\csname PY@tok@ni\endcsname{\let\PY@bf=\textbf\def\PY@tc##1{\textcolor[rgb]{0.60,0.60,0.60}{##1}}}
\expandafter\def\csname PY@tok@na\endcsname{\def\PY@tc##1{\textcolor[rgb]{0.49,0.56,0.16}{##1}}}
\expandafter\def\csname PY@tok@nt\endcsname{\let\PY@bf=\textbf\def\PY@tc##1{\textcolor[rgb]{0.00,0.50,0.00}{##1}}}
\expandafter\def\csname PY@tok@nd\endcsname{\def\PY@tc##1{\textcolor[rgb]{0.67,0.13,1.00}{##1}}}
\expandafter\def\csname PY@tok@s\endcsname{\def\PY@tc##1{\textcolor[rgb]{0.73,0.13,0.13}{##1}}}
\expandafter\def\csname PY@tok@sd\endcsname{\let\PY@it=\textit\def\PY@tc##1{\textcolor[rgb]{0.73,0.13,0.13}{##1}}}
\expandafter\def\csname PY@tok@si\endcsname{\let\PY@bf=\textbf\def\PY@tc##1{\textcolor[rgb]{0.73,0.40,0.53}{##1}}}
\expandafter\def\csname PY@tok@se\endcsname{\let\PY@bf=\textbf\def\PY@tc##1{\textcolor[rgb]{0.73,0.40,0.13}{##1}}}
\expandafter\def\csname PY@tok@sr\endcsname{\def\PY@tc##1{\textcolor[rgb]{0.73,0.40,0.53}{##1}}}
\expandafter\def\csname PY@tok@ss\endcsname{\def\PY@tc##1{\textcolor[rgb]{0.10,0.09,0.49}{##1}}}
\expandafter\def\csname PY@tok@sx\endcsname{\def\PY@tc##1{\textcolor[rgb]{0.00,0.50,0.00}{##1}}}
\expandafter\def\csname PY@tok@m\endcsname{\def\PY@tc##1{\textcolor[rgb]{0.40,0.40,0.40}{##1}}}
\expandafter\def\csname PY@tok@gh\endcsname{\let\PY@bf=\textbf\def\PY@tc##1{\textcolor[rgb]{0.00,0.00,0.50}{##1}}}
\expandafter\def\csname PY@tok@gu\endcsname{\let\PY@bf=\textbf\def\PY@tc##1{\textcolor[rgb]{0.50,0.00,0.50}{##1}}}
\expandafter\def\csname PY@tok@gd\endcsname{\def\PY@tc##1{\textcolor[rgb]{0.63,0.00,0.00}{##1}}}
\expandafter\def\csname PY@tok@gi\endcsname{\def\PY@tc##1{\textcolor[rgb]{0.00,0.63,0.00}{##1}}}
\expandafter\def\csname PY@tok@gr\endcsname{\def\PY@tc##1{\textcolor[rgb]{1.00,0.00,0.00}{##1}}}
\expandafter\def\csname PY@tok@ge\endcsname{\let\PY@it=\textit}
\expandafter\def\csname PY@tok@gs\endcsname{\let\PY@bf=\textbf}
\expandafter\def\csname PY@tok@gp\endcsname{\let\PY@bf=\textbf\def\PY@tc##1{\textcolor[rgb]{0.00,0.00,0.50}{##1}}}
\expandafter\def\csname PY@tok@go\endcsname{\def\PY@tc##1{\textcolor[rgb]{0.53,0.53,0.53}{##1}}}
\expandafter\def\csname PY@tok@gt\endcsname{\def\PY@tc##1{\textcolor[rgb]{0.00,0.27,0.87}{##1}}}
\expandafter\def\csname PY@tok@err\endcsname{\def\PY@bc##1{\setlength{\fboxsep}{0pt}\fcolorbox[rgb]{1.00,0.00,0.00}{1,1,1}{\strut ##1}}}
\expandafter\def\csname PY@tok@kc\endcsname{\let\PY@bf=\textbf\def\PY@tc##1{\textcolor[rgb]{0.00,0.50,0.00}{##1}}}
\expandafter\def\csname PY@tok@kd\endcsname{\let\PY@bf=\textbf\def\PY@tc##1{\textcolor[rgb]{0.00,0.50,0.00}{##1}}}
\expandafter\def\csname PY@tok@kn\endcsname{\let\PY@bf=\textbf\def\PY@tc##1{\textcolor[rgb]{0.00,0.50,0.00}{##1}}}
\expandafter\def\csname PY@tok@kr\endcsname{\let\PY@bf=\textbf\def\PY@tc##1{\textcolor[rgb]{0.00,0.50,0.00}{##1}}}
\expandafter\def\csname PY@tok@bp\endcsname{\def\PY@tc##1{\textcolor[rgb]{0.00,0.50,0.00}{##1}}}
\expandafter\def\csname PY@tok@fm\endcsname{\def\PY@tc##1{\textcolor[rgb]{0.00,0.00,1.00}{##1}}}
\expandafter\def\csname PY@tok@vc\endcsname{\def\PY@tc##1{\textcolor[rgb]{0.10,0.09,0.49}{##1}}}
\expandafter\def\csname PY@tok@vg\endcsname{\def\PY@tc##1{\textcolor[rgb]{0.10,0.09,0.49}{##1}}}
\expandafter\def\csname PY@tok@vi\endcsname{\def\PY@tc##1{\textcolor[rgb]{0.10,0.09,0.49}{##1}}}
\expandafter\def\csname PY@tok@vm\endcsname{\def\PY@tc##1{\textcolor[rgb]{0.10,0.09,0.49}{##1}}}
\expandafter\def\csname PY@tok@sa\endcsname{\def\PY@tc##1{\textcolor[rgb]{0.73,0.13,0.13}{##1}}}
\expandafter\def\csname PY@tok@sb\endcsname{\def\PY@tc##1{\textcolor[rgb]{0.73,0.13,0.13}{##1}}}
\expandafter\def\csname PY@tok@sc\endcsname{\def\PY@tc##1{\textcolor[rgb]{0.73,0.13,0.13}{##1}}}
\expandafter\def\csname PY@tok@dl\endcsname{\def\PY@tc##1{\textcolor[rgb]{0.73,0.13,0.13}{##1}}}
\expandafter\def\csname PY@tok@s2\endcsname{\def\PY@tc##1{\textcolor[rgb]{0.73,0.13,0.13}{##1}}}
\expandafter\def\csname PY@tok@sh\endcsname{\def\PY@tc##1{\textcolor[rgb]{0.73,0.13,0.13}{##1}}}
\expandafter\def\csname PY@tok@s1\endcsname{\def\PY@tc##1{\textcolor[rgb]{0.73,0.13,0.13}{##1}}}
\expandafter\def\csname PY@tok@mb\endcsname{\def\PY@tc##1{\textcolor[rgb]{0.40,0.40,0.40}{##1}}}
\expandafter\def\csname PY@tok@mf\endcsname{\def\PY@tc##1{\textcolor[rgb]{0.40,0.40,0.40}{##1}}}
\expandafter\def\csname PY@tok@mh\endcsname{\def\PY@tc##1{\textcolor[rgb]{0.40,0.40,0.40}{##1}}}
\expandafter\def\csname PY@tok@mi\endcsname{\def\PY@tc##1{\textcolor[rgb]{0.40,0.40,0.40}{##1}}}
\expandafter\def\csname PY@tok@il\endcsname{\def\PY@tc##1{\textcolor[rgb]{0.40,0.40,0.40}{##1}}}
\expandafter\def\csname PY@tok@mo\endcsname{\def\PY@tc##1{\textcolor[rgb]{0.40,0.40,0.40}{##1}}}
\expandafter\def\csname PY@tok@ch\endcsname{\let\PY@it=\textit\def\PY@tc##1{\textcolor[rgb]{0.25,0.50,0.50}{##1}}}
\expandafter\def\csname PY@tok@cm\endcsname{\let\PY@it=\textit\def\PY@tc##1{\textcolor[rgb]{0.25,0.50,0.50}{##1}}}
\expandafter\def\csname PY@tok@cpf\endcsname{\let\PY@it=\textit\def\PY@tc##1{\textcolor[rgb]{0.25,0.50,0.50}{##1}}}
\expandafter\def\csname PY@tok@c1\endcsname{\let\PY@it=\textit\def\PY@tc##1{\textcolor[rgb]{0.25,0.50,0.50}{##1}}}
\expandafter\def\csname PY@tok@cs\endcsname{\let\PY@it=\textit\def\PY@tc##1{\textcolor[rgb]{0.25,0.50,0.50}{##1}}}

\def\PYZbs{\char`\\}
\def\PYZus{\char`\_}
\def\PYZob{\char`\{}
\def\PYZcb{\char`\}}
\def\PYZca{\char`\^}
\def\PYZam{\char`\&}
\def\PYZlt{\char`\<}
\def\PYZgt{\char`\>}
\def\PYZsh{\char`\#}
\def\PYZpc{\char`\%}
\def\PYZdl{\char`\$}
\def\PYZhy{\char`\-}
\def\PYZsq{\char`\'}
\def\PYZdq{\char`\"}
\def\PYZti{\char`\~}
% for compatibility with earlier versions
\def\PYZat{@}
\def\PYZlb{[}
\def\PYZrb{]}
\makeatother


    % Exact colors from NB
    \definecolor{incolor}{rgb}{0.0, 0.0, 0.5}
    \definecolor{outcolor}{rgb}{0.545, 0.0, 0.0}



    
    % Prevent overflowing lines due to hard-to-break entities
    \sloppy 
    % Setup hyperref package
    \hypersetup{
      breaklinks=true,  % so long urls are correctly broken across lines
      colorlinks=true,
      urlcolor=urlcolor,
      linkcolor=linkcolor,
      citecolor=citecolor,
      }
    % Slightly bigger margins than the latex defaults
    
    \geometry{verbose,tmargin=1in,bmargin=1in,lmargin=1in,rmargin=1in}
    
    

    \begin{document}
    
    
    \maketitle
    
    

    
    \hypertarget{project-finding-lane-lines-on-the-road}{%
\section{\texorpdfstring{Project: \textbf{Finding Lane Lines on the
Road}}{Project: Finding Lane Lines on the Road}}\label{project-finding-lane-lines-on-the-road}}

\begin{center}\rule{0.5\linewidth}{\linethickness}\end{center}

Develop a pipeline to identify lane lines on the road. You must apply it
on a series of individual images, provided in the \emph{test\_images}
folder.

Once you have a result that looks roughly like the image
\emph{line-segments-example} in the examples folder (also shown below),
you'll need to try to average and/or extrapolate the line segments
you've detected to map out the full extent of the lane lines.

    \textbf{The tools you have are color selection, region of interest
selection, grayscaling, Gaussian smoothing, Canny Edge Detection and
Hough Tranform line detection. You are also free to explore and try
other techniques that were not presented. Your goal is piece together a
pipeline to detect the line segments in the image, then
average/extrapolate them and draw them onto the image for display (as
below).}

\begin{center}\rule{0.5\linewidth}{\linethickness}\end{center}

Your output should look something like this (above) after detecting line
segments using the helper functions below

Your goal is to connect/average/extrapolate line segments to get output
like this

    \hypertarget{import-packages}{%
\subsection{Import Packages}\label{import-packages}}

    \begin{Verbatim}[commandchars=\\\{\}]
{\color{incolor}In [{\color{incolor}2}]:} \PY{c+c1}{\PYZsh{}importing some useful packages}
        \PY{k+kn}{import} \PY{n+nn}{matplotlib}\PY{n+nn}{.}\PY{n+nn}{pyplot} \PY{k}{as} \PY{n+nn}{plt}
        \PY{k+kn}{import} \PY{n+nn}{numpy} \PY{k}{as} \PY{n+nn}{np}
        \PY{k+kn}{import} \PY{n+nn}{cv2}
        \PY{o}{\PYZpc{}}\PY{k}{matplotlib} inline
\end{Verbatim}


    \hypertarget{read-in-an-image}{%
\subsection{Read in an Image}\label{read-in-an-image}}

    \begin{Verbatim}[commandchars=\\\{\}]
{\color{incolor}In [{\color{incolor}3}]:} \PY{c+c1}{\PYZsh{}reading in an image}
        \PY{n}{image} \PY{o}{=} \PY{n}{cv2}\PY{o}{.}\PY{n}{imread}\PY{p}{(}\PY{l+s+s1}{\PYZsq{}}\PY{l+s+s1}{test\PYZus{}images/solidWhiteRight.jpg}\PY{l+s+s1}{\PYZsq{}}\PY{p}{)}
        \PY{n}{image} \PY{o}{=} \PY{n}{cv2}\PY{o}{.}\PY{n}{cvtColor}\PY{p}{(}\PY{n}{image}\PY{p}{,} \PY{n}{cv2}\PY{o}{.}\PY{n}{COLOR\PYZus{}BGR2RGB}\PY{p}{)}
        
        \PY{c+c1}{\PYZsh{}printing out some stats and plotting}
        \PY{n+nb}{print}\PY{p}{(}\PY{l+s+s1}{\PYZsq{}}\PY{l+s+s1}{This image is:}\PY{l+s+s1}{\PYZsq{}}\PY{p}{,} \PY{n+nb}{type}\PY{p}{(}\PY{n}{image}\PY{p}{)}\PY{p}{,} \PY{l+s+s1}{\PYZsq{}}\PY{l+s+s1}{with dimensions:}\PY{l+s+s1}{\PYZsq{}}\PY{p}{,} \PY{n}{image}\PY{o}{.}\PY{n}{shape}\PY{p}{)}
        \PY{n}{plt}\PY{o}{.}\PY{n}{imshow}\PY{p}{(}\PY{n}{image}\PY{p}{)}  \PY{c+c1}{\PYZsh{} if you wanted to show a single color channel image called \PYZsq{}gray\PYZsq{}, for example, call as plt.imshow(gray, cmap=\PYZsq{}gray\PYZsq{})}
\end{Verbatim}


    \begin{Verbatim}[commandchars=\\\{\}]
This image is: <class 'numpy.ndarray'> with dimensions: (540, 960, 3)

    \end{Verbatim}

\begin{Verbatim}[commandchars=\\\{\}]
{\color{outcolor}Out[{\color{outcolor}3}]:} <matplotlib.image.AxesImage at 0x7f764af13eb8>
\end{Verbatim}
            
    \begin{center}
    \adjustimage{max size={0.9\linewidth}{0.9\paperheight}}{output_5_2.png}
    \end{center}
    { \hspace*{\fill} \\}
    
    \hypertarget{ideas-for-lane-detection-pipeline}{%
\subsection{Ideas for Lane Detection
Pipeline}\label{ideas-for-lane-detection-pipeline}}

    \textbf{Some OpenCV functions that might be useful for this project
are:}

\texttt{cv2.inRange()} for color selection\\
\texttt{cv2.fillPoly()} for regions selection\\
\texttt{cv2.line()} to draw lines on an image given endpoints\\
\texttt{cv2.addWeighted()} to coadd / overlay two images
\texttt{cv2.cvtColor()} to grayscale or change color
\texttt{cv2.imwrite()} to output images to file\\
\texttt{cv2.bitwise\_and()} to apply a mask to an image

    \hypertarget{helper-functions}{%
\subsection{Helper Functions}\label{helper-functions}}

    Below are some helper functions to help get you started.

    \begin{Verbatim}[commandchars=\\\{\}]
{\color{incolor}In [{\color{incolor}213}]:} \PY{k+kn}{import} \PY{n+nn}{math}
          
          \PY{k}{def} \PY{n+nf}{grayscale}\PY{p}{(}\PY{n}{img}\PY{p}{)}\PY{p}{:}
              \PY{l+s+sd}{\PYZdq{}\PYZdq{}\PYZdq{}Applies the Grayscale transform}
          \PY{l+s+sd}{    This will return an image with only one color channel}
          \PY{l+s+sd}{    but NOTE: to see the returned image as grayscale}
          \PY{l+s+sd}{    (assuming your grayscaled image is called \PYZsq{}gray\PYZsq{})}
          \PY{l+s+sd}{    you should call plt.imshow(gray, cmap=\PYZsq{}gray\PYZsq{})\PYZdq{}\PYZdq{}\PYZdq{}}
              \PY{k}{return} \PY{n}{cv2}\PY{o}{.}\PY{n}{cvtColor}\PY{p}{(}\PY{n}{img}\PY{p}{,} \PY{n}{cv2}\PY{o}{.}\PY{n}{COLOR\PYZus{}RGB2GRAY}\PY{p}{)}
              \PY{c+c1}{\PYZsh{} Or use BGR2GRAY if you read an image with cv2.imread()}
              \PY{c+c1}{\PYZsh{} return cv2.cvtColor(img, cv2.COLOR\PYZus{}BGR2GRAY)}
              
          \PY{k}{def} \PY{n+nf}{canny}\PY{p}{(}\PY{n}{img}\PY{p}{,} \PY{n}{low\PYZus{}threshold}\PY{p}{,} \PY{n}{high\PYZus{}threshold}\PY{p}{)}\PY{p}{:}
              \PY{l+s+sd}{\PYZdq{}\PYZdq{}\PYZdq{}Applies the Canny transform\PYZdq{}\PYZdq{}\PYZdq{}}
              \PY{k}{return} \PY{n}{cv2}\PY{o}{.}\PY{n}{Canny}\PY{p}{(}\PY{n}{img}\PY{p}{,} \PY{n}{low\PYZus{}threshold}\PY{p}{,} \PY{n}{high\PYZus{}threshold}\PY{p}{)}
          
          \PY{k}{def} \PY{n+nf}{gaussian\PYZus{}blur}\PY{p}{(}\PY{n}{img}\PY{p}{,} \PY{n}{kernel\PYZus{}size}\PY{p}{)}\PY{p}{:}
              \PY{l+s+sd}{\PYZdq{}\PYZdq{}\PYZdq{}Applies a Gaussian Noise kernel\PYZdq{}\PYZdq{}\PYZdq{}}
              \PY{k}{return} \PY{n}{cv2}\PY{o}{.}\PY{n}{GaussianBlur}\PY{p}{(}\PY{n}{img}\PY{p}{,} \PY{p}{(}\PY{n}{kernel\PYZus{}size}\PY{p}{,} \PY{n}{kernel\PYZus{}size}\PY{p}{)}\PY{p}{,} \PY{l+m+mi}{0}\PY{p}{)}
          
          \PY{k}{def} \PY{n+nf}{region\PYZus{}of\PYZus{}interest}\PY{p}{(}\PY{n}{img}\PY{p}{,} \PY{n}{vertices}\PY{p}{)}\PY{p}{:}
              \PY{l+s+sd}{\PYZdq{}\PYZdq{}\PYZdq{}}
          \PY{l+s+sd}{    Applies an image mask.}
          \PY{l+s+sd}{    }
          \PY{l+s+sd}{    Only keeps the region of the image defined by the polygon}
          \PY{l+s+sd}{    formed from `vertices`. The rest of the image is set to black.}
          \PY{l+s+sd}{    `vertices` should be a numpy array of integer points.}
          \PY{l+s+sd}{    \PYZdq{}\PYZdq{}\PYZdq{}}
              \PY{c+c1}{\PYZsh{}defining a blank mask to start with}
              \PY{n}{mask} \PY{o}{=} \PY{n}{np}\PY{o}{.}\PY{n}{zeros\PYZus{}like}\PY{p}{(}\PY{n}{img}\PY{p}{)}   
              
              \PY{c+c1}{\PYZsh{}defining a 3 channel or 1 channel color to fill the mask with depending on the input image}
              \PY{k}{if} \PY{n+nb}{len}\PY{p}{(}\PY{n}{img}\PY{o}{.}\PY{n}{shape}\PY{p}{)} \PY{o}{\PYZgt{}} \PY{l+m+mi}{2}\PY{p}{:}
                  \PY{n}{channel\PYZus{}count} \PY{o}{=} \PY{n}{img}\PY{o}{.}\PY{n}{shape}\PY{p}{[}\PY{l+m+mi}{2}\PY{p}{]}  \PY{c+c1}{\PYZsh{} i.e. 3 or 4 depending on your image}
                  \PY{n}{ignore\PYZus{}mask\PYZus{}color} \PY{o}{=} \PY{p}{(}\PY{l+m+mi}{255}\PY{p}{,}\PY{p}{)} \PY{o}{*} \PY{n}{channel\PYZus{}count}
              \PY{k}{else}\PY{p}{:}
                  \PY{n}{ignore\PYZus{}mask\PYZus{}color} \PY{o}{=} \PY{l+m+mi}{255}
                  
              \PY{c+c1}{\PYZsh{}filling pixels inside the polygon defined by \PYZdq{}vertices\PYZdq{} with the fill color    }
              \PY{n}{cv2}\PY{o}{.}\PY{n}{fillPoly}\PY{p}{(}\PY{n}{mask}\PY{p}{,} \PY{n}{vertices}\PY{p}{,} \PY{n}{ignore\PYZus{}mask\PYZus{}color}\PY{p}{)}
              
              \PY{c+c1}{\PYZsh{}returning the image only where mask pixels are nonzero}
              \PY{n}{masked\PYZus{}image} \PY{o}{=} \PY{n}{cv2}\PY{o}{.}\PY{n}{bitwise\PYZus{}and}\PY{p}{(}\PY{n}{img}\PY{p}{,} \PY{n}{mask}\PY{p}{)}
              \PY{k}{return} \PY{n}{masked\PYZus{}image}
          
          
          \PY{k}{def} \PY{n+nf}{draw\PYZus{}lines}\PY{p}{(}\PY{n}{img}\PY{p}{,} \PY{n}{lines}\PY{p}{,} \PY{n}{color}\PY{o}{=}\PY{p}{[}\PY{l+m+mi}{255}\PY{p}{,} \PY{l+m+mi}{0}\PY{p}{,} \PY{l+m+mi}{0}\PY{p}{]}\PY{p}{,} \PY{n}{thickness}\PY{o}{=}\PY{l+m+mi}{10}\PY{p}{)}\PY{p}{:}
              \PY{l+s+sd}{\PYZdq{}\PYZdq{}\PYZdq{}}
          \PY{l+s+sd}{    This function draws `lines` with `color` and `thickness`.    }
          \PY{l+s+sd}{    Lines are drawn on the image inplace (mutates the image).}
          \PY{l+s+sd}{    If you want to make the lines semi\PYZhy{}transparent, think about combining}
          \PY{l+s+sd}{    this function with the weighted\PYZus{}img() function below}
          \PY{l+s+sd}{    \PYZdq{}\PYZdq{}\PYZdq{}}
              
              \PY{n}{right} \PY{o}{=} \PY{l+m+mi}{0}
              
              \PY{n}{left} \PY{o}{=} \PY{l+m+mi}{0}
              
              \PY{k}{for} \PY{n}{line} \PY{o+ow}{in} \PY{n}{lines}\PY{p}{:}
                  \PY{k}{for} \PY{n}{x1}\PY{p}{,}\PY{n}{y1}\PY{p}{,}\PY{n}{x2}\PY{p}{,}\PY{n}{y2} \PY{o+ow}{in} \PY{n}{line}\PY{p}{:}
                      
                      \PY{n}{m} \PY{o}{=} \PY{p}{(}\PY{n}{y2} \PY{o}{\PYZhy{}} \PY{n}{y1}\PY{p}{)} \PY{o}{/} \PY{p}{(}\PY{n}{x2} \PY{o}{\PYZhy{}} \PY{n}{x1}\PY{p}{)} 
                      
                      \PY{k}{if} \PY{n}{m} \PY{o}{\PYZgt{}} \PY{l+m+mf}{0.5} \PY{o+ow}{and} \PY{n}{m} \PY{o}{\PYZlt{}} \PY{l+m+mf}{0.7} \PY{o+ow}{and} \PY{n}{right} \PY{o}{==} \PY{l+m+mi}{0}\PY{p}{:}    
                          \PY{n}{ex1\PYZus{}x} \PY{o}{=} \PY{l+m+mi}{500}
                          \PY{n}{ex2\PYZus{}y} \PY{o}{=} \PY{l+m+mi}{540}
          
                          \PY{n}{ex1\PYZus{}y} \PY{o}{=} \PY{n+nb}{int}\PY{p}{(}\PY{n}{y1}\PY{o}{+}\PY{p}{(}\PY{p}{(}\PY{n}{ex1\PYZus{}x}\PY{o}{\PYZhy{}}\PY{n}{x1}\PY{p}{)}\PY{o}{*}\PY{n}{m}\PY{p}{)}\PY{p}{)}
          
                          \PY{n}{ex2\PYZus{}x} \PY{o}{=} \PY{n+nb}{int}\PY{p}{(}\PY{p}{(}\PY{p}{(}\PY{n}{y2}\PY{o}{\PYZhy{}}\PY{n}{ex2\PYZus{}y}\PY{p}{)}\PY{o}{/}\PY{n}{m}\PY{p}{)}\PY{o}{\PYZhy{}}\PY{n}{x2}\PY{p}{)}\PY{o}{*}\PY{p}{(}\PY{o}{\PYZhy{}}\PY{l+m+mi}{1}\PY{p}{)}
          
                          \PY{n}{cv2}\PY{o}{.}\PY{n}{line}\PY{p}{(}\PY{n}{img}\PY{p}{,} \PY{p}{(}\PY{n}{ex2\PYZus{}x}\PY{p}{,} \PY{n}{ex2\PYZus{}y}\PY{p}{)}\PY{p}{,} \PY{p}{(}\PY{n}{ex1\PYZus{}x}\PY{p}{,} \PY{n}{ex1\PYZus{}y}\PY{p}{)}\PY{p}{,} \PY{n}{color}\PY{p}{,} \PY{n}{thickness}\PY{p}{)}
                          
                          \PY{n}{right} \PY{o}{=} \PY{l+m+mi}{1}
          
                      \PY{k}{elif} \PY{n}{m} \PY{o}{\PYZlt{}} \PY{o}{\PYZhy{}}\PY{l+m+mf}{0.75} \PY{o+ow}{and} \PY{n}{m} \PY{o}{\PYZgt{}} \PY{o}{\PYZhy{}}\PY{l+m+mf}{0.85} \PY{o+ow}{and} \PY{n}{left} \PY{o}{==} \PY{l+m+mi}{0}\PY{p}{:} 
                          
                          \PY{n}{ex1\PYZus{}x} \PY{o}{=} \PY{l+m+mi}{460}
                          \PY{n}{ex2\PYZus{}y} \PY{o}{=} \PY{l+m+mi}{540}
          
                          \PY{n}{ex1\PYZus{}y} \PY{o}{=} \PY{n+nb}{int}\PY{p}{(}\PY{n}{y1}\PY{o}{+}\PY{p}{(}\PY{p}{(}\PY{n}{ex1\PYZus{}x}\PY{o}{\PYZhy{}}\PY{n}{x1}\PY{p}{)}\PY{o}{*}\PY{n}{m}\PY{p}{)}\PY{p}{)}
          
                          \PY{n}{ex2\PYZus{}x} \PY{o}{=} \PY{n+nb}{int}\PY{p}{(}\PY{p}{(}\PY{p}{(}\PY{n}{y2}\PY{o}{\PYZhy{}}\PY{n}{ex2\PYZus{}y}\PY{p}{)}\PY{o}{/}\PY{n}{m}\PY{p}{)}\PY{o}{\PYZhy{}}\PY{n}{x2}\PY{p}{)}\PY{o}{*}\PY{p}{(}\PY{o}{\PYZhy{}}\PY{l+m+mi}{1}\PY{p}{)}
          
                          \PY{n}{cv2}\PY{o}{.}\PY{n}{line}\PY{p}{(}\PY{n}{img}\PY{p}{,} \PY{p}{(}\PY{n}{ex2\PYZus{}x}\PY{p}{,} \PY{n}{ex2\PYZus{}y}\PY{p}{)}\PY{p}{,} \PY{p}{(}\PY{n}{ex1\PYZus{}x}\PY{p}{,} \PY{n}{ex1\PYZus{}y}\PY{p}{)}\PY{p}{,} \PY{n}{color}\PY{p}{,} \PY{n}{thickness}\PY{p}{)}  
                          
                          \PY{n}{left} \PY{o}{=} \PY{l+m+mi}{1}
                  
          \PY{k}{def} \PY{n+nf}{hough\PYZus{}lines}\PY{p}{(}\PY{n}{img}\PY{p}{,} \PY{n}{rho}\PY{p}{,} \PY{n}{theta}\PY{p}{,} \PY{n}{threshold}\PY{p}{,} \PY{n}{min\PYZus{}line\PYZus{}len}\PY{p}{,} \PY{n}{max\PYZus{}line\PYZus{}gap}\PY{p}{)}\PY{p}{:}
              \PY{l+s+sd}{\PYZdq{}\PYZdq{}\PYZdq{}}
          \PY{l+s+sd}{    `img` should be the output of a Canny transform.}
          \PY{l+s+sd}{        }
          \PY{l+s+sd}{    Returns an image with hough lines drawn.}
          \PY{l+s+sd}{    \PYZdq{}\PYZdq{}\PYZdq{}}
              \PY{n}{lines} \PY{o}{=} \PY{n}{cv2}\PY{o}{.}\PY{n}{HoughLinesP}\PY{p}{(}\PY{n}{img}\PY{p}{,} \PY{n}{rho}\PY{p}{,} \PY{n}{theta}\PY{p}{,} \PY{n}{threshold}\PY{p}{,} \PY{n}{np}\PY{o}{.}\PY{n}{array}\PY{p}{(}\PY{p}{[}\PY{p}{]}\PY{p}{)}\PY{p}{,} \PY{n}{minLineLength}\PY{o}{=}\PY{n}{min\PYZus{}line\PYZus{}len}\PY{p}{,} \PY{n}{maxLineGap}\PY{o}{=}\PY{n}{max\PYZus{}line\PYZus{}gap}\PY{p}{)}
              \PY{n}{line\PYZus{}img} \PY{o}{=} \PY{n}{np}\PY{o}{.}\PY{n}{zeros}\PY{p}{(}\PY{p}{(}\PY{n}{img}\PY{o}{.}\PY{n}{shape}\PY{p}{[}\PY{l+m+mi}{0}\PY{p}{]}\PY{p}{,} \PY{n}{img}\PY{o}{.}\PY{n}{shape}\PY{p}{[}\PY{l+m+mi}{1}\PY{p}{]}\PY{p}{,} \PY{l+m+mi}{3}\PY{p}{)}\PY{p}{,} \PY{n}{dtype}\PY{o}{=}\PY{n}{np}\PY{o}{.}\PY{n}{uint8}\PY{p}{)}
              \PY{n}{draw\PYZus{}lines}\PY{p}{(}\PY{n}{line\PYZus{}img}\PY{p}{,} \PY{n}{lines}\PY{p}{)}
              \PY{k}{return} \PY{n}{line\PYZus{}img}
          
          \PY{k}{def} \PY{n+nf}{weighted\PYZus{}img}\PY{p}{(}\PY{n}{img}\PY{p}{,} \PY{n}{initial\PYZus{}img}\PY{p}{,} \PY{n}{alpha}\PY{o}{=}\PY{l+m+mf}{0.8}\PY{p}{,} \PY{n}{beta}\PY{o}{=}\PY{l+m+mf}{1.}\PY{p}{,} \PY{n}{gamma}\PY{o}{=}\PY{l+m+mf}{0.}\PY{p}{)}\PY{p}{:}
              \PY{l+s+sd}{\PYZdq{}\PYZdq{}\PYZdq{}}
          \PY{l+s+sd}{    `img` is the output of the hough\PYZus{}lines(), An image with lines drawn on it.}
          \PY{l+s+sd}{    Should be a blank image (all black) with lines drawn on it.}
          \PY{l+s+sd}{    }
          \PY{l+s+sd}{    `initial\PYZus{}img` should be the image before any processing.}
          \PY{l+s+sd}{    }
          \PY{l+s+sd}{    The result image is computed as follows:}
          \PY{l+s+sd}{    }
          \PY{l+s+sd}{    initial\PYZus{}img * α + img * β + γ}
          \PY{l+s+sd}{    NOTE: initial\PYZus{}img and img must be the same shape!}
          \PY{l+s+sd}{    \PYZdq{}\PYZdq{}\PYZdq{}}
              \PY{k}{return} \PY{n}{cv2}\PY{o}{.}\PY{n}{addWeighted}\PY{p}{(}\PY{n}{initial\PYZus{}img}\PY{p}{,} \PY{n}{alpha}\PY{p}{,} \PY{n}{img}\PY{p}{,} \PY{n}{beta}\PY{p}{,} \PY{n}{gamma}\PY{p}{)}
\end{Verbatim}


    \hypertarget{test-images}{%
\subsection{Test Images}\label{test-images}}

Build your pipeline to work on the images in the directory
``test\_images''

    \begin{Verbatim}[commandchars=\\\{\}]
{\color{incolor}In [{\color{incolor}214}]:} \PY{k+kn}{import} \PY{n+nn}{os}
          
          \PY{n}{path} \PY{o}{=} \PY{l+s+s2}{\PYZdq{}}\PY{l+s+s2}{test\PYZus{}images/}\PY{l+s+s2}{\PYZdq{}}
          \PY{n}{files} \PY{o}{=} \PY{n}{os}\PY{o}{.}\PY{n}{listdir}\PY{p}{(}\PY{n}{path}\PY{p}{)}
          \PY{n}{images} \PY{o}{=} \PY{p}{[}\PY{p}{]}
          
          \PY{n}{f}\PY{p}{,} \PY{n}{plots} \PY{o}{=} \PY{n}{plt}\PY{o}{.}\PY{n}{subplots}\PY{p}{(}\PY{p}{(}\PY{n+nb}{len}\PY{p}{(}\PY{n}{files}\PY{p}{)}\PY{o}{+}\PY{l+m+mi}{3}\PY{o}{\PYZhy{}}\PY{l+m+mi}{1}\PY{p}{)}\PY{o}{/}\PY{o}{/}\PY{l+m+mi}{3}\PY{p}{,} \PY{l+m+mi}{3}\PY{p}{,} \PY{n}{figsize}\PY{o}{=}\PY{p}{(}\PY{l+m+mi}{20}\PY{p}{,}\PY{l+m+mi}{10}\PY{p}{)}\PY{p}{)}
          \PY{n}{plots} \PY{o}{=} \PY{p}{[}\PY{n}{plot} \PY{k}{for} \PY{n}{sublist} \PY{o+ow}{in} \PY{n}{plots} \PY{k}{for} \PY{n}{plot} \PY{o+ow}{in} \PY{n}{sublist}\PY{p}{]}
          
          \PY{k}{for} \PY{n}{file}\PY{p}{,} \PY{n}{plot} \PY{o+ow}{in} \PY{n+nb}{zip}\PY{p}{(}\PY{n}{files}\PY{p}{,} \PY{n}{plots}\PY{p}{)}\PY{p}{:}
              \PY{n}{image} \PY{o}{=} \PY{n}{cv2}\PY{o}{.}\PY{n}{cvtColor}\PY{p}{(}\PY{n}{cv2}\PY{o}{.}\PY{n}{imread}\PY{p}{(}\PY{n}{os}\PY{o}{.}\PY{n}{path}\PY{o}{.}\PY{n}{join}\PY{p}{(}\PY{n}{path}\PY{p}{,} \PY{n}{file}\PY{p}{)}\PY{p}{)}\PY{p}{,} \PY{n}{cv2}\PY{o}{.}\PY{n}{COLOR\PYZus{}BGR2RGB}\PY{p}{)}
              \PY{n}{plot}\PY{o}{.}\PY{n}{set\PYZus{}title}\PY{p}{(}\PY{n}{file}\PY{p}{)}
              \PY{n}{plot}\PY{o}{.}\PY{n}{imshow}\PY{p}{(}\PY{n}{image}\PY{p}{)}
              \PY{n}{images}\PY{o}{.}\PY{n}{append}\PY{p}{(}\PY{p}{(}\PY{n}{image}\PY{p}{,} \PY{n}{file}\PY{p}{)}\PY{p}{)}
\end{Verbatim}


    \begin{center}
    \adjustimage{max size={0.9\linewidth}{0.9\paperheight}}{output_12_0.png}
    \end{center}
    { \hspace*{\fill} \\}
    
    \hypertarget{build-a-lane-finding-pipeline}{%
\subsection{Build a Lane Finding
Pipeline}\label{build-a-lane-finding-pipeline}}

    Build the pipeline and run your solution on all test\_images.

Try tuning the various parameters, especially the low and high Canny
thresholds as well as the Hough lines parameters.

    \begin{Verbatim}[commandchars=\\\{\}]
{\color{incolor}In [{\color{incolor}215}]:} \PY{c+c1}{\PYZsh{} TODO: Build your pipeline that will draw lane lines segments on the test\PYZus{}images}
          
          \PY{n}{path} \PY{o}{=} \PY{l+s+s2}{\PYZdq{}}\PY{l+s+s2}{test\PYZus{}images/}\PY{l+s+s2}{\PYZdq{}}
          \PY{n}{files} \PY{o}{=} \PY{n}{os}\PY{o}{.}\PY{n}{listdir}\PY{p}{(}\PY{n}{path}\PY{p}{)}
          \PY{n}{images\PYZus{}lines} \PY{o}{=} \PY{p}{[}\PY{p}{]}
          
          \PY{n}{vertices} \PY{o}{=} \PY{n}{np}\PY{o}{.}\PY{n}{array}\PY{p}{(}\PY{p}{[}\PY{p}{[}\PY{l+m+mi}{0}\PY{p}{,}\PY{l+m+mi}{540}\PY{p}{]}\PY{p}{,}\PY{p}{[}\PY{l+m+mi}{460}\PY{p}{,}\PY{l+m+mi}{320}\PY{p}{]}\PY{p}{,}\PY{p}{[}\PY{l+m+mi}{500}\PY{p}{,}\PY{l+m+mi}{320}\PY{p}{]}\PY{p}{,}\PY{p}{[}\PY{l+m+mi}{960}\PY{p}{,}\PY{l+m+mi}{540}\PY{p}{]}\PY{p}{]}\PY{p}{)}
          
          \PY{n}{f}\PY{p}{,} \PY{n}{plots} \PY{o}{=} \PY{n}{plt}\PY{o}{.}\PY{n}{subplots}\PY{p}{(}\PY{p}{(}\PY{n+nb}{len}\PY{p}{(}\PY{n}{files}\PY{p}{)}\PY{o}{+}\PY{l+m+mi}{3}\PY{o}{\PYZhy{}}\PY{l+m+mi}{1}\PY{p}{)}\PY{o}{/}\PY{o}{/}\PY{l+m+mi}{3}\PY{p}{,} \PY{l+m+mi}{3}\PY{p}{,} \PY{n}{figsize}\PY{o}{=}\PY{p}{(}\PY{l+m+mi}{20}\PY{p}{,}\PY{l+m+mi}{10}\PY{p}{)}\PY{p}{)}
          \PY{n}{plots} \PY{o}{=} \PY{p}{[}\PY{n}{plot} \PY{k}{for} \PY{n}{sublist} \PY{o+ow}{in} \PY{n}{plots} \PY{k}{for} \PY{n}{plot} \PY{o+ow}{in} \PY{n}{sublist}\PY{p}{]}
          
          \PY{k}{for} \PY{n}{file}\PY{p}{,} \PY{n}{plot} \PY{o+ow}{in} \PY{n+nb}{zip}\PY{p}{(}\PY{n}{files}\PY{p}{,} \PY{n}{plots}\PY{p}{)}\PY{p}{:}
              
              \PY{n}{test} \PY{o}{=} \PY{n}{cv2}\PY{o}{.}\PY{n}{cvtColor}\PY{p}{(}\PY{n}{cv2}\PY{o}{.}\PY{n}{imread}\PY{p}{(}\PY{n}{os}\PY{o}{.}\PY{n}{path}\PY{o}{.}\PY{n}{join}\PY{p}{(}\PY{n}{path}\PY{p}{,} \PY{n}{file}\PY{p}{)}\PY{p}{)}\PY{p}{,} \PY{n}{cv2}\PY{o}{.}\PY{n}{COLOR\PYZus{}BGR2RGB}\PY{p}{)}
          
              \PY{c+c1}{\PYZsh{}test = cv2.imread(\PYZsq{}test\PYZus{}images/solidYellowCurve2.jpg\PYZsq{})}
              
              \PY{c+c1}{\PYZsh{} Region mask}
              
              \PY{n}{region\PYZus{}mask} \PY{o}{=} \PY{n}{region\PYZus{}of\PYZus{}interest}\PY{p}{(}\PY{n}{test}\PY{p}{,} \PY{p}{[}\PY{n}{vertices}\PY{p}{]}\PY{p}{)}
          
          
              \PY{n}{lower\PYZus{}lines} \PY{o}{=} \PY{n}{np}\PY{o}{.}\PY{n}{array}\PY{p}{(}\PY{p}{[}\PY{l+m+mi}{210}\PY{p}{,}\PY{l+m+mi}{150}\PY{p}{,}\PY{l+m+mi}{0}\PY{p}{]}\PY{p}{)}
              \PY{n}{upper\PYZus{}lines} \PY{o}{=} \PY{n}{np}\PY{o}{.}\PY{n}{array}\PY{p}{(}\PY{p}{[}\PY{l+m+mi}{255}\PY{p}{,}\PY{l+m+mi}{255}\PY{p}{,}\PY{l+m+mi}{255}\PY{p}{]}\PY{p}{)}
          
          
              \PY{c+c1}{\PYZsh{} Define the masked area}
              \PY{n}{mask\PYZus{}lines} \PY{o}{=} \PY{n}{cv2}\PY{o}{.}\PY{n}{inRange}\PY{p}{(}\PY{n}{region\PYZus{}mask}\PY{p}{,} \PY{n}{lower\PYZus{}lines}\PY{p}{,} \PY{n}{upper\PYZus{}lines}\PY{p}{)}
          
              \PY{c+c1}{\PYZsh{} Apply Blurr}
          
              \PY{n}{blur\PYZus{}image} \PY{o}{=} \PY{n}{gaussian\PYZus{}blur}\PY{p}{(}\PY{n}{mask\PYZus{}lines}\PY{p}{,} \PY{l+m+mi}{9}\PY{p}{)}
          
              \PY{c+c1}{\PYZsh{} Canny parmeters}
              \PY{n}{low} \PY{o}{=} \PY{l+m+mi}{50}
              \PY{n}{high} \PY{o}{=} \PY{l+m+mi}{100}
          
              \PY{n}{image\PYZus{}edges} \PY{o}{=} \PY{n}{canny}\PY{p}{(}\PY{n}{blur\PYZus{}image}\PY{p}{,} \PY{n}{low}\PY{p}{,} \PY{n}{high}\PY{p}{)}
              
              \PY{c+c1}{\PYZsh{}second blur}
              
              \PY{n}{second\PYZus{}blur\PYZus{}image} \PY{o}{=} \PY{n}{gaussian\PYZus{}blur}\PY{p}{(}\PY{n}{image\PYZus{}edges}\PY{p}{,} \PY{l+m+mi}{9}\PY{p}{)}
          
              \PY{c+c1}{\PYZsh{} define the Hough transform parameters}
              \PY{n}{rho} \PY{o}{=} \PY{l+m+mi}{5}
              \PY{n}{theta} \PY{o}{=} \PY{n}{np}\PY{o}{.}\PY{n}{pi}\PY{o}{/}\PY{l+m+mi}{180}
              \PY{n}{threshold} \PY{o}{=} \PY{l+m+mi}{130}
              \PY{n}{min\PYZus{}line\PYZus{}len} \PY{o}{=} \PY{l+m+mi}{10}
              \PY{n}{max\PYZus{}line\PYZus{}gap} \PY{o}{=} \PY{l+m+mi}{20}
          
              \PY{n}{lines} \PY{o}{=} \PY{n}{hough\PYZus{}lines}\PY{p}{(}\PY{n}{second\PYZus{}blur\PYZus{}image}\PY{p}{,} \PY{n}{rho}\PY{p}{,} \PY{n}{theta}\PY{p}{,} \PY{n}{threshold}\PY{p}{,} \PY{n}{min\PYZus{}line\PYZus{}len}\PY{p}{,} \PY{n}{max\PYZus{}line\PYZus{}gap}\PY{p}{)}
              
              \PY{n}{plot}\PY{o}{.}\PY{n}{set\PYZus{}title}\PY{p}{(}\PY{n}{file}\PY{p}{)}
              \PY{n}{plot}\PY{o}{.}\PY{n}{imshow}\PY{p}{(}\PY{n}{lines}\PY{p}{,} \PY{n}{cmap} \PY{o}{=} \PY{l+s+s2}{\PYZdq{}}\PY{l+s+s2}{gray}\PY{l+s+s2}{\PYZdq{}}\PY{p}{)}
              \PY{n}{images\PYZus{}lines}\PY{o}{.}\PY{n}{append}\PY{p}{(}\PY{p}{(}\PY{n}{lines}\PY{p}{,} \PY{n}{file}\PY{p}{)}\PY{p}{)}
\end{Verbatim}


    \begin{center}
    \adjustimage{max size={0.9\linewidth}{0.9\paperheight}}{output_15_0.png}
    \end{center}
    { \hspace*{\fill} \\}
    
    \hypertarget{improve-the-draw_lines-function}{%
\subsection{Improve the draw\_lines()
function}\label{improve-the-draw_lines-function}}

\textbf{At this point, you should have the Hough line segments drawn
onto the road. Extend your code to define a line to run the full length
of the visible lane based on the line segments you identified with the
Hough Transform. Try to average and/or extrapolate the line segments
you've detected to map out the full extent of the lane lines. The output
should draw a single, solid line over the left lane line and a single,
solid line over the right lane line. The lines should start from the
bottom of the image and extend out to the top of the region of
interest.}

    \begin{Verbatim}[commandchars=\\\{\}]
{\color{incolor}In [{\color{incolor}216}]:} \PY{c+c1}{\PYZsh{} TODO: Build your pipeline that will draw complete lane lines on the test\PYZus{}images}
          \PY{n}{f}\PY{p}{,} \PY{n}{plots} \PY{o}{=} \PY{n}{plt}\PY{o}{.}\PY{n}{subplots}\PY{p}{(}\PY{p}{(}\PY{n+nb}{len}\PY{p}{(}\PY{n}{files}\PY{p}{)}\PY{o}{+}\PY{l+m+mi}{3}\PY{o}{\PYZhy{}}\PY{l+m+mi}{1}\PY{p}{)}\PY{o}{/}\PY{o}{/}\PY{l+m+mi}{3}\PY{p}{,} \PY{l+m+mi}{3}\PY{p}{,} \PY{n}{figsize}\PY{o}{=}\PY{p}{(}\PY{l+m+mi}{20}\PY{p}{,}\PY{l+m+mi}{10}\PY{p}{)}\PY{p}{)}
          \PY{n}{plots} \PY{o}{=} \PY{p}{[}\PY{n}{plot} \PY{k}{for} \PY{n}{sublist} \PY{o+ow}{in} \PY{n}{plots} \PY{k}{for} \PY{n}{plot} \PY{o+ow}{in} \PY{n}{sublist}\PY{p}{]}
          
          \PY{k}{for} \PY{n}{img\PYZus{}lines}\PY{p}{,}\PY{n}{img}\PY{p}{,} \PY{n}{plot} \PY{o+ow}{in} \PY{n+nb}{zip}\PY{p}{(}\PY{n}{images\PYZus{}lines}\PY{p}{,}\PY{n}{images}\PY{p}{,} \PY{n}{plots}\PY{p}{)}\PY{p}{:}
              
              \PY{n}{completed} \PY{o}{=} \PY{n}{weighted\PYZus{}img}\PY{p}{(}\PY{n}{img\PYZus{}lines}\PY{p}{[}\PY{l+m+mi}{0}\PY{p}{]}\PY{p}{,} \PY{n}{img}\PY{p}{[}\PY{l+m+mi}{0}\PY{p}{]}\PY{p}{)}
              
              \PY{n}{plot}\PY{o}{.}\PY{n}{set\PYZus{}title}\PY{p}{(}\PY{n}{file}\PY{p}{)}
              \PY{n}{plot}\PY{o}{.}\PY{n}{imshow}\PY{p}{(}\PY{n}{completed}\PY{p}{,} \PY{n}{cmap} \PY{o}{=} \PY{l+s+s2}{\PYZdq{}}\PY{l+s+s2}{gray}\PY{l+s+s2}{\PYZdq{}}\PY{p}{)}
              
              
\end{Verbatim}


    \begin{center}
    \adjustimage{max size={0.9\linewidth}{0.9\paperheight}}{output_17_0.png}
    \end{center}
    { \hspace*{\fill} \\}
    

    % Add a bibliography block to the postdoc
    
    
    
    \end{document}
